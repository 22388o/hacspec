\documentclass[11pt,a4paper]{article}

\usepackage[T1]{fontenc}
\usepackage[utf8]{inputenc}
\usepackage[english]{babel}
\usepackage{fullpage}
\usepackage{lmodern}
\usepackage{amsmath,amssymb}
\usepackage{mathpartir}
\usepackage{xcolor}
\usepackage{booktabs,longtable}
\usepackage{csquotes}

\newcommand{\todo}[2]{\textcolor{purple!50!orange}{[#1]: #2}}

\newcommand{\rustspec}{\textsf{Rustspec}}

\title{\rustspec{} formalization}

\author{Denis Merigoux}



%%%%%%%%%%%%%%%%%%% Syntax
\newcommand{\synvar}[1]{\ensuremath{#1}}
\newcommand{\syntext}[1]{\texttt{#1}}
\newcommand{\synkeyword}[1]{\textcolor{red!60!black}{\syntext{#1}}}
\newcommand{\synpunct}[1]{\textcolor{black!40!white}{\texttt{#1}}}

%% Keywords
\newcommand{\synuse}{\synkeyword{use}\;}
\newcommand{\synconst}{\synkeyword{const}\;}
\newcommand{\syntypalias}{\synkeyword{type}\;}
\newcommand{\synfn}{\synkeyword{fn}\;}
\newcommand{\synbool}{\synkeyword{bool}}
\newcommand{\synint}{\synkeyword{int}}
\newcommand{\synunitt}{\synkeyword{unit}}
\newcommand{\synseq}{\syntext{Seq}}
\newcommand{\syncopy}{\syntext{Copy}}
\newcommand{\synclone}{\syntext{clone()}}
\newcommand{\synlet}{\synkeyword{let}\;}
\newcommand{\synmut}{\synkeyword{mut}\;}
\newcommand{\synif}{\synkeyword{if}\;}
\newcommand{\synthen}{\;\synkeyword{then}\;}
\newcommand{\synelse}{\;\synkeyword{else}\;}
\newcommand{\synfor}{\synkeyword{for}\;}
\newcommand{\synin}{\;\synkeyword{in}\;}
\newcommand{\syntrue}{\synkeyword{true}}
\newcommand{\synfalse}{\synkeyword{false}}

%%Macros
\newcommand{\synarraymacro}{\synkeyword{array!}}
\newcommand{\synpolymacro}{\synkeyword{poly!}}

%% Punctucation
\newcommand{\synsc}{\synpunct{;}}
\newcommand{\syntyped}{\;\synpunct{:}\;}
\newcommand{\syneq}{\;\synpunct{=}\;}
\newcommand{\synlparen}{\synpunct{(}\;}
\newcommand{\synrparen}{\;\synpunct{)}}
\newcommand{\syncomma}{\synpunct{,}\;}
\newcommand{\syndot}{\synpunct{.}}
\newcommand{\synref}{\synpunct{\&}}
\newcommand{\synand}{\;\synpunct{\&\&}\;}
\newcommand{\synor}{\;\synpunct{||}\;}
\newcommand{\synxor}{\;\synpunct{\^}\;}
\newcommand{\synadd}{\;\synpunct{+}\;}
\newcommand{\synminus}{\;\synpunct{-}\;}
\newcommand{\synmult}{\;\synpunct{*}\;}
\newcommand{\syndiv}{\;\synpunct{/}\;}
\newcommand{\synnot}{\synpunct{\~}\;}
\newcommand{\synlangle}{\synpunct{<}\;}
\newcommand{\synrangle}{\;\synpunct{>}}
\newcommand{\synlbracket}{\synpunct{\{}\;}
\newcommand{\synrbracket}{\;\synpunct{\}}}
\newcommand{\synarrow}{\;\synpunct{->}\;}
\newcommand{\synrange}{\;\synpunct{..}\;}
\newcommand{\synlsquare}{\synpunct{[}\;}
\newcommand{\synrsquare}{\;\synpunct{]}}
\newcommand{\synellipsis}{\synpunct{,\ldots,}\;}
\newcommand{\synscellipsis}{\synpunct{;\ldots;}\;}
\newcommand{\synunit}{\synpunct{()}}

%% BNF
\newcommand{\syndef}{$::=$}
\newcommand{\synalt}{\;$|$\;}

%%%%%%%%%%%%%%%%%%%% Typing

\newcommand{\typctx}[1]{\textcolor{green!50!black}{\ensuremath{#1}}}

\newcommand{\typempty}{\typctx{\varnothing}}
\newcommand{\typtyped}{\;\typctx{:}\;}
\newcommand{\typsc}{\typctx{;}\;}
\newcommand{\typcomma}{\typctx{,}\;}
\newcommand{\typarrow}{\typctx{\;\rightarrow}\;}
\newcommand{\typlparen}{\typctx{(}\;}
\newcommand{\typrparen}{\;\typctx{)}}
\newcommand{\typlsquare}{\typctx{[}\;}
\newcommand{\typrsquare}{\;\typctx{]}}
\newcommand{\typlangle}{\typctx{<}\;}
\newcommand{\typrangle}{\;\typctx{>}}
\newcommand{\typeq}{\;\typctx{=}\;}
\newcommand{\typcomp}{\;\typctx{\circ}\;}
\newcommand{\typellipsis}{\typctx{,\ldots,}\;}
\newcommand{\typderive}{\;\typctx{\vdash}\;}
\newcommand{\typsym}{\;\typctx{\sim}\;}
\newcommand{\typproduce}{\;\typctx{\Rrightarrow}\;}


\begin{document}
\maketitle

\section{Introduction}

\rustspec{} is a subset of Rust designed for cryptographic specification. It is the successor of
Hacspec \cite{karthik2018hacspec}. Indeed, the strong typing environment of Rust and its low-level
focus are more suitable to cryptographic specification than the subset of Python that was used for
Hacspec. The main features of \rustspec{} are :
\begin{description}
  \item[Pure functions] By restricting borrowing and making full use of the \syntext{Copy} trait, \rustspec{} feels like functional programming.
  \item[Primitive array support] Array manipulation is the cornerstone of cryptographic code, hence
  \rustspec{} arrays have primitive support in the language and a dedicated manipulation library.
  \item[Integers] \rustspec{} supports all kinds of integers : word-sized integers (\synkeyword{u8}, \synkeyword{u32}, etc.), secret integers (constant-time by type abstraction design),
  natural integers (for field arithmetic).
\end{description}

\rustspec{} is concretely implemented as a Rust crate containing the types and functions of the
language. However, \rustspec{} programs should also fall into a strict subset of Rust, that this
document formalizes.

The \rustspec{} language, being embedded in Rust, shares with it some linearity properties of its
type system. However, we severely restrict the borrowing feature: while retaining enough expressive
power to write cryptographic specifications, the language behaves much simpler. The approach is conceptually similar to \cite{radanne2019kindly}, although \rustspec{} is less expressive.

The formalization presented here does not cover all the features of the \rustspec{} library. It
focuses on the core of the language, its interesting type system and execution semantics. More
precisely, we did not include the following elements :
\begin{itemize}
  \item anything that can easily be desugared like assignment operators (\synpunct{+=}, etc);
  \item the \synif{} expression, that coexists with the \synif{} statement;
  \item all the various types of integers, the casts between them and all the operators to
  manipulate them (we only have basic \synint{} type with arithmetic);
  \item polymorphism and traits, as we assume that the Rust compiler does the job of monomorphizing
  everything for us.
\end{itemize}

\section{Syntax}

The \rustspec{} language is a subset of Rust. Apart from the basic \synint{} and \synbool{} types,
the language operates on arrays of fixed size, known at compilation (\synarraymacro{}) or not
(\synseq). The biggest limitation compared to the full Rust language is the borrowing patterns
that we allow. Indeed, we restrict borrowing to only immutable borrowing (\synref{}),
and we also restrict where this borrowing can happen: only in function arguments or at function call
sites. Additionally, functions cannot return types containing references. Because of these restrictions,
there is not much you can do with types subject to linearity in \rustspec, compared to what you
can do in Rust. However, we argue that it is enough for cryptographic specifications.

\begin{center}
\begin{longtable}{lrrll}
Items&\synvar{i}&\syndef&\synarraymacro\synlparen\synvar{t}\syncomma\synvar{\mu}\syncomma $n\in\mathbb{N}$\synrparen\synsc&array type declaration\\
&&\synalt&\synfn\synvar{f}\synlparen$[$\synvar{d}$]^+$\synrparen\synarrow\synvar{\mu}\;\synlbracket$[$\synvar{s}$]^+$\synrbracket&function declaration\\
Argument declaration&\synvar{d}&\syndef&\synvar{x}\syntyped\synvar{\tau}&\\
Reference-free type&\synvar{\mu}&\syndef&\synunitt\synalt\synbool\synalt\synint&base types\\
&&\synalt&\synseq\synlangle\synvar{\mu}\synrangle&sequence\\
&&\synalt&\synvar{t}&type variable\\
&&\synalt&\synlparen$[$\synvar{\mu}$]^+$\synrparen&tuple\\
Type&\synvar{\tau}&\syndef&\synvar{\mu}&base types\\
&&\synalt&\synref\synvar{\mu}&immutable reference\\
Statement&\synvar{s}&\syndef&\synlet$($\synmut$)$\;\synvar{x}\syntyped\synvar{\tau}\syneq\synvar{e}&let binding\\
&&\synalt&\synvar{x}\syneq\synvar{e}&mutable variable reassignment\\
&&\synalt&\synif\synvar{e}\synthen\synlbracket$[$\synvar{s}$]^+$\synrbracket\;$($\synelse\synlbracket$[$\synvar{s}$]^+$\synrbracket$)$&conditional statements\\
&&\synalt&\synfor\synvar{x}\synin\synvar{e}\synrange\synvar{e}\;\synlbracket$[$\synvar{s}$]^+$\synrbracket&for loop (integers only)\\
&&\synalt&\synvar{e}\synlsquare\synvar{e}\synrsquare\syneq\synvar{e}&array update\\
&&\synalt&\synvar{e}&return expression\\
Expression&\synvar{e}&\syndef&\synunit\synalt\syntrue\synalt\synfalse&unit and boolean literals\\
&&\synalt&$n\in\mathbb{N}$&integer literal\\
&&\synalt&\synvar{x}&variable\\
&&\synalt&\synvar{f}\synlparen$[$\synvar{a}$]^+$\synrparen&function call\\
&&\synalt&\synvar{e}\;\synvar{\odot}\;\synvar{e}&binary operations\\
&&\synalt&\synvar{\oslash}\;\synvar{e}&unary operations\\
&&\synalt&\synlparen $[$\synvar{e}$]^+$\synrparen&tuple constructor\\
&&\synalt&\synvar{e}\syndot$(n\in\mathbb{N})$&tuple field access\\
&&\synalt&\synvar{e}\synlsquare\synvar{e}\synrsquare&array or seq index\\
Function argument&\synvar{a}&\syndef&\synvar{e}&linear argument\\
&&\synalt&\synref\synvar{e}&call-site borrowing\\
Binary operators&\synvar{\odot}&\syndef&\synadd\synalt\synminus\synalt\synmult\synalt\syndiv\synalt\synand\synalt\synor\\
Unary operators&\synvar{\oslash}&\syndef&\synminus\synalt\synnot&
\end{longtable}
\end{center}

\section{Typing}

The typing environnement of \rustspec{} is fairly standard. We need a type dictionary to enforce
the named type discipline of Rust that covers the types declared by the \synarraymacro{} macro.

\begin{center}
\begin{longtable}{lrrll}
Typing context&\typctx{\Gamma}&\syndef&\typempty&empty context\\
&&\synalt& \synvar{x}\typtyped\synvar{\tau}\typcomma\typctx{\Gamma}&variable\\
&&\synalt& \synvar{f}\typtyped\typlparen$[$\synvar{\tau}$]^+$\typrparen\typarrow\synvar{\mu}\typcomma\typctx{\Gamma}&function\\
Type dictionary&\typctx{\Delta}&\syndef&\typempty&empty dictionnary\\
&&\synalt&\synvar{t}\typarrow\typlsquare\synvar{\mu}\typsc $n\in\mathbb{N}$\typrsquare\typcomma\typctx{\Delta}&array type\\
\end{longtable}
\end{center}

The restrictions on borrowing lead to severe limitations on how we can manipulate values
of linear type in our language, rendering it quite useless at first sight. Indeed, when you receive
a reference as a function argument, you can only use it in expressions and perform identity let
bindings with it. You cannot store it in memory or in a tuple and pass it around indirectly
in your program. This behavior is well-suited for input and output buffers in cryptographic code.

However, in the spirit of Rust, we introduce an escape hatch from linearity under the form of the \syncopy{} trait
implementation. This trait, that is primitive to the Rust language, is used to distinguish the values
that are \enquote{cheap} to copy. This concerns all the reference-free \synvar{\mu} types except \synseq, whose size is not known at compilation time (and thus can be arbitrarily large).
Paradoxically, \synarraymacro{} types that can be as large as their \synseq{} counterparts do benefit
from the \syncopy{} trait; we replicate here the behavior of Rust. Indeed, because
the lenght is known at compilation time, the code generation backend of Rust (LLVM) can optimize the
representation of the array in memory, especially if the size is small.

With this setup, both \synarraymacro{} and \synseq{} represent a table of data. The moral difference between
them is that \synarraymacro{} is a table passed by value, whereas \synseq{} is a tabled passed by
reference (immutable). In the cryptographic specifications, \synseq{} is mostly used for input
and output messages, whose length is known only at runtime. Fixed-size chunks or blocks of data
are rather implemented using \synarraymacro{}.

\begin{center}
\begin{longtable}{lc}\toprule
Implementing the \syncopy{} trait&\typctx{\Delta}\typderive\synvar{\tau}\typtyped\syncopy
\\\bottomrule
\end{longtable}
\begin{mathpar}
\inferrule[CopyUnit]{
{}
}{
\typctx{\Delta}\typderive\synunitt\typtyped\syncopy
}

\inferrule[CopyBool]{
{}
}{
\typctx{\Delta}\typderive\synbool\typtyped\syncopy
}

\inferrule[CopyInt]{
{}
}{
\typctx{\Delta}\typderive\synint\typtyped\syncopy
}

\inferrule[CopyTuple]{
\typctx{\Delta}\typderive\synvar{\tau_1}\typtyped\syncopy\\
\cdots\\
\typctx{\Delta}\typderive\synvar{\tau_n}\typtyped\syncopy
}{
\typctx{\Delta}\typderive\typlparen\synvar{\tau_1}\typellipsis\synvar{\tau_n}\typrparen\typtyped\syncopy
}

\inferrule[CopyArray]{
\typctx{\Delta}\typderive\synvar{\mu}\typtyped\syncopy
}{
\synvar{t}\typarrow\typlsquare\synvar{\mu}\typsc n\typrsquare\typcomma\typctx{\Delta}\typderive\synvar{t}\typtyped\syncopy
}
\end{mathpar}
\end{center}

Because Rust has an affine type system, \rustspec{} also enjoys an affine typing context with
associated splitting rules (\TirName{SplitLinear}). Please note that immutable references values can be duplicated freely
in the context (\TirName{SplitDuplicable}). During an elaboration phase inside the Rust compiler, the linearity
of the type system gets circumvented for \syncopy{} values with the insertion of \synclone{} functions
call that perform a copy of the value wherever the linear type system forces a copy of the value to
be made. We formalize this behavior here by allowing \syncopy{} types duplication in the typing
context, like immutable references (\TirName{SplitCopy}). Lastly, functions are always duplicable in the context (\TirName{SplitFunction}).

\begin{center}
\begin{longtable}{lc}\toprule
Context splitting&\typctx{\Delta}\typderive\typctx{\Gamma}\typeq\typctx{\Gamma_1}\typcomp\typctx{\Gamma_2}
\\\bottomrule
\end{longtable}
\begin{mathpar}
\inferrule[SplitEmpty]{
{}
}{
\typctx{\Delta}\typderive\typempty\typeq\typempty\typcomp\typempty
}

\inferrule[SplitLinear1]{
\typctx{\Delta}\typderive\typctx{\Gamma}\typeq\typctx{\Gamma_1}\typcomp\typctx{\Gamma_2}
}{
\typctx{\Delta}\typderive\synvar{x}\typtyped\synvar{\tau}\typcomma\typctx{\Gamma}\typeq
\typlparen\synvar{x}\typtyped\synvar{\tau}\typcomma\typctx{\Gamma_1}\typrparen\typcomp
\typctx{\Gamma_2}
}

\inferrule[SplitLinear2]{
\typctx{\Delta}\typderive\typctx{\Gamma}\typeq\typctx{\Gamma_1}\typcomp\typctx{\Gamma_2}
}{
\typctx{\Delta}\typderive\synvar{x}\typtyped\synvar{\tau}\typcomma\typctx{\Gamma}\typeq
\typctx{\Gamma_1}\typcomp
\typlparen\synvar{x}\typtyped\synvar{\tau}\typcomma\typctx{\Gamma_2}\typrparen
}

\inferrule[SplitDuplicable]{
\typctx{\Delta}\typderive\typctx{\Gamma}\typeq\typctx{\Gamma_1}\typcomp\typctx{\Gamma_2}
}{
\typctx{\Delta}\typderive\synvar{x}\typtyped\synref\synvar{\tau}\typcomma\typctx{\Gamma}\typeq
\typlparen\synvar{x}\typtyped\synref\synvar{\tau}\typcomma\typctx{\Gamma_1}\typrparen\typcomp
\typlparen\synvar{x}\typtyped\synref\synvar{\tau}\typcomma\typctx{\Gamma_2}\typrparen
}

\inferrule[SplitCopy]{
\typctx{\Delta}\typderive\typctx{\Gamma}\typeq\typctx{\Gamma_1}\typcomp\typctx{\Gamma_2}\\
\typctx{\Delta}\typderive\synvar{\tau}\typtyped\syncopy
}{
\typctx{\Delta}\typderive\synvar{x}\typtyped\synvar{\tau}\typcomma\typctx{\Gamma}\typeq
\typlparen\synvar{x}\typtyped\synvar{\tau}\typcomma\typctx{\Gamma_1}\typrparen\typcomp
\typlparen\synvar{x}\typtyped\synvar{\tau}\typcomma\typctx{\Gamma_2}\typrparen
}

\inferrule[SplitFunction]{
\typctx{\Delta}\typderive\typctx{\Gamma}\typeq\typctx{\Gamma_1}\typcomp\typctx{\Gamma_2}
}{
\typctx{\Delta}\typderive\synvar{f}\typtyped\typlparen\typctx{\mu_1}\typellipsis\typctx{\mu_n}\typrparen\typarrow\synvar{\tau}\typcomma\typctx{\Gamma}\typeq
\typlparen\synvar{f}\typtyped\typlparen\typctx{\mu_1}\typellipsis\typctx{\mu_n}\typrparen\typarrow\synvar{\tau}\typcomma\typctx{\Gamma_1}\typrparen\typcomp
\typlparen\synvar{f}\typtyped\typlparen\typctx{\mu_1}\typellipsis\typctx{\mu_n}\typrparen\typarrow\synvar{\tau}\typcomma\typctx{\Gamma_2}\typrparen
}
\end{mathpar}
\end{center}

We can now proceed to the main typing judgments.
\TirName{TypVarLinear} and \TirName{TypVarDup} reflect the variable typing present in the context.
\TirName{TypTupleCons} only allows non-reference values inside a tuple, with a linear context
splitting to check each term of the tuple. \TirName{TypArrayAccess}, \TirName{TypSeqAccess} and
\TirName{TypSeqRefAccess} specify the array indexing syntax, which is overloaded to work with
both \synarraymacro{}, \synseq{} and \synref\synseq{}. This corresponds to the implementing of
the \syntext{Index} trait in Rust.

The function call rule, \TirName{TypFuncCall}, is the most complex rule of the typing judgment,
because it contains the
restricted borrowing form allowed in \rustspec{}. First, note that the context is split for
typechecking the arguments of the function, because a linear value cannot be used in two arguments.
Next, \TirName{TypFunArgLin} ensures that the arguments are well-typed. However, if an argument
is borrowed at call-site, then \TirName{TypFunArgDup} checks the value that is being borrowed
under the reference. In our degenerate pattern of borrowing, \TirName{TypFunArgDup} is doing the
work of the Rust borrow checker. The last rules for binary and unary operations are standard.

\begin{center}
\begin{longtable}{lc}\toprule
Expression typing&\typctx{\Gamma}\typsc\typctx{\Delta}\typderive\synvar{e}\typtyped\synvar{\tau}\\
Function argument typing&\typctx{\Gamma}\typsc\typctx{\Delta}\typderive\synvar{a}\typsym\synvar{\tau}
\\\bottomrule
\end{longtable}
\begin{mathpar}
\inferrule[TypUnit]{
{}
}{
\typctx{\Gamma}\typsc\typctx{\Delta}\typderive\synunit\typtyped\synunitt
}

\inferrule[TypTrue]{
{}
}{
\typctx{\Gamma}\typsc\typctx{\Delta}\typderive\syntrue\typtyped\synbool
}

\inferrule[TypFalse]{
{}
}{
\typctx{\Gamma}\typsc\typctx{\Delta}\typderive\synfalse\typtyped\synbool
}

\inferrule[TypInt]{n\in\mathbb{N}}{
\typctx{\Gamma}\typsc\typctx{\Delta}\typderive n\typtyped\synint
}
\end{mathpar}
\begin{mathpar}
\inferrule[TypVarLinear]{
{}
}{
\synvar{x}\typtyped\synvar{\tau}\typcomma\typctx{\Gamma}\typsc\typctx{\Delta}\typderive\synvar{x}\typtyped\synvar{\tau}
}

\inferrule[TypVarDup]{
{}
}{
\synvar{x}\typtyped\synref\synvar{\tau}\typcomma\typctx{\Gamma}\typsc\typctx{\Delta}\typderive\synvar{x}\typtyped\synref\synvar{\tau}
}
\end{mathpar}
\begin{mathpar}
\inferrule[TypTupleCons]{
\typctx{\Delta}\typderive\typctx{\Gamma}\typeq\typctx{\Gamma_1}\typcomp\cdots\typcomp\typctx{\Gamma_{n}}\\
\forall i\in[\![1,n]\!],\;\typctx{\Gamma}\typsc\typctx{\Delta}\typderive\synvar{e_i}\typtyped\synvar{\mu_i}
}{
\typctx{\Gamma}\typsc\typctx{\Delta}\typderive \synlparen \synvar{e_1}\synellipsis\synvar{e_n}\synrparen\typtyped\synlparen\synvar{\mu_1}\synellipsis\synvar{\mu_n}\synrparen
}
\end{mathpar}
\begin{mathpar}
\inferrule[TypTupleAccess]{
\typctx{\Gamma}\typsc\typctx{\Delta}\typderive\synvar{e}\typtyped\synlparen\synvar{\mu_1}\synellipsis\synvar{\mu_n}\synrparen\\
i\in[\![1,n]\!]
}{
\typctx{\Gamma}\typsc\typctx{\Delta}\typderive\synvar{e}\syndot i\typtyped \synvar{\mu_i}
}
\end{mathpar}
\begin{mathpar}
\inferrule[TypArrayAccess]{
\typctx{\Gamma}\typsc\synvar{t}\typarrow\typlsquare\synvar{\mu}\typsc n\typrsquare\typcomma\typctx{\Delta}\typderive\synvar{e_1}\typtyped\synvar{t}\\
\typctx{\Gamma}\typsc\synvar{t}\typarrow\typlsquare\synvar{\mu}\typsc n\typrsquare\typcomma\typctx{\Delta}\typderive\synvar{e_2}\typtyped\synint
}{
\typctx{\Gamma}\typsc\synvar{t}\typarrow\typlsquare\synvar{\mu}\typsc n\typrsquare\typcomma\typctx{\Delta}\typderive\synvar{e_1}\synlsquare\synvar{e_2}\synrsquare\typtyped\synvar{\mu}
}

\inferrule[TypSeqAccess]{
\typctx{\Gamma}\typsc\typctx{\Delta}\typderive\synvar{e_1}\typtyped\synseq\synlangle\synvar{\mu}\synrangle\\
\typctx{\Gamma}\typsc\typctx{\Delta}\typderive\synvar{e_2}\typtyped\synint
}{
\typctx{\Gamma}\typsc\typctx{\Delta}\typderive\synvar{e_1}\synlsquare\synvar{e_2}\synrsquare\typtyped\synvar{\mu}
}

\inferrule[TypSeqRefAccess]{
\typctx{\Gamma}\typsc\typctx{\Delta}\typderive\synvar{e_1}\typtyped\synref\synseq\synlangle\synvar{\mu}\synrangle\\
\typctx{\Gamma}\typsc\typctx{\Delta}\typderive\synvar{e_2}\typtyped\synint
}{
\typctx{\Gamma}\typsc\typctx{\Delta}\typderive\synvar{e_1}\synlsquare\synvar{e_2}\synrsquare\typtyped\synvar{\mu}
}
\end{mathpar}
\begin{mathpar}
\inferrule[TypFuncCall]{
\typctx{\Gamma}\typeq\synvar{f}\typtyped\typlparen\typctx{\mu_1}\typellipsis\typctx{\mu_n}\typrparen\typarrow\synvar{\tau}\typcomma\typctx{\Gamma'}\\
\typctx{\Delta}\typderive\typctx{\Gamma}\typeq\typctx{\Gamma_1}\typcomp\cdots\typcomp\typctx{\Gamma_{n}}\\
\forall i\in[\![1,n]\!],\;\typctx{\Gamma_i}\typsc\typctx{\Delta}\typderive\synvar{a_i}\typsym\typctx{\mu_i}\\
}{
\typctx{\Gamma}\typsc\typctx{\Delta}\typderive\synvar{f}\synlparen\synvar{a_1}\synellipsis\synvar{a_n}\synrparen\typtyped\synvar{\tau}
}
\end{mathpar}
\begin{mathpar}
\inferrule[TypFunArgLin]{
\typctx{\Gamma}\typsc\typctx{\Delta}\typderive\synvar{e}\typtyped\synvar{\tau}
}{
\typctx{\Gamma}\typsc\typctx{\Delta}\typderive\synvar{e}\typsym\synvar{\tau}
}

\inferrule[TypFunArgDup]{
\typctx{\Gamma}\typsc\typctx{\Delta}\typderive\synvar{e}\typtyped\synvar{\mu}
}{
\typctx{\Gamma}\typsc\typctx{\Delta}\typderive\synref\synvar{e}\typsym\synref\synvar{\mu}
}
\end{mathpar}
\begin{mathpar}
\inferrule[TypBinopInt]{
\typctx{\Gamma}\typsc\typctx{\Delta}\typderive\synvar{e_1}\typtyped\synvar{\synint}\\
\typctx{\Gamma}\typsc\typctx{\Delta}\typderive\synvar{e_2}\typtyped\synvar{\synint}\\
\synvar{\odot}\in\{\synadd,\synminus,\synmult,\syndiv\}
}{
\typctx{\Gamma}\typsc\typctx{\Delta}\typderive\synvar{e_1}\;\synvar{\odot}\;\synvar{e_2}\typtyped\synvar{\synint}
}
\end{mathpar}
\begin{mathpar}
\inferrule[TypBinopBool]{
\typctx{\Gamma}\typsc\typctx{\Delta}\typderive\synvar{e_1}\typtyped\synvar{\synbool}\\
\typctx{\Gamma}\typsc\typctx{\Delta}\typderive\synvar{e_2}\typtyped\synvar{\synbool}\\
\synvar{\odot}\in\{\synand,\synor\}
}{
\typctx{\Gamma}\typsc\typctx{\Delta}\typderive\synvar{e_1}\;\synvar{\odot}\;\synvar{e_2}\typtyped\synvar{\synbool}
}

\inferrule[TypUnopInt]{
\typctx{\Gamma}\typsc\typctx{\Delta}\typderive\synvar{e}\typtyped\synint\\
\oslash\in\{\synminus\}
}{
\typctx{\Gamma}\typsc\typctx{\Delta}\typderive\synvar{\oslash}\;\synvar{e}\typtyped\synint
}
\end{mathpar}
\begin{mathpar}
\inferrule[TypUnopBool]{
\typctx{\Gamma}\typsc\typctx{\Delta}\typderive\synvar{e}\typtyped\synbool\\
\oslash\in\{\text{\synnot}\}
}{
\typctx{\Gamma}\typsc\typctx{\Delta}\typderive\synvar{\oslash}\;\synvar{e}\typtyped\synbool
}
\end{mathpar}
\end{center}


Let's now move to the statement typing. We've chosen statements here rather than nested expressions
because of the Rust behavior of the \synif statement and the \synfor loop. A list of statement corresponds
to a block, introduced in Rust by \synlbracket $\cdots\!$ \synrbracket.
The single statement typing judgment produces a new \typctx{\Gamma'} because of variable
definitions inside a block. Single statements also yield back a type, which is \synunitt most of the
time. This statement type is used because the last statement of the function is also the return
value of the function.

While \TirName{TypLet} is very classic, \TirName{TypLetMut} introduces a new mutable local variable,
that can later be reassigned (\TirName{TypReassign}) in the program.
The main use of this feature is for variables that are
mutated inside a \synfor loop. Indeed, because \synfor loops are restricted to integer range iteration,
we cannot express what would normally be a fold without these mutable variables. Because the mutable
variables are local to a block, we do not need to formalize a full-fledged heap for the operational
semantics. Rather, we will model them as a limited piece of state that gets passed around during
execution.

Next, \TirName{TypArrayAssign} and \TirName{TypSeqAssign} define the overloading of the array
update syntax that works for both \synarraymacro{} and \synseq{}. Note that \TirName{TypIfThenElse}
use the same context \typctx{\Gamma} for the two branches of the conditional.

\begin{center}
\begin{longtable}{lc}\toprule
Statement typing&\typctx{\Gamma}\typsc\typctx{\Delta}\typderive\synvar{s}\typtyped\synvar{\tau}\typproduce\typctx{\Gamma'}\\
Statement sequence typing&\typctx{\Gamma}\typsc\typctx{\Delta}\typderive\synvar{s_1}\synscellipsis\synvar{s_n}\typtyped\synvar{\tau}
\\\bottomrule
\end{longtable}
\begin{mathpar}
\inferrule[TypLet]{
\typctx{\Gamma}\typsc{\typctx{\Delta}}\typderive\synvar{e}\typtyped\synvar{\tau}
}{
\typctx{\Gamma}\typsc\typctx{\Delta}\typderive\synlet\synvar{x}\syntyped\synvar{\tau}\syneq\synvar{e}\typtyped\synunitt\typproduce\typctx{\Gamma}\typcomma\synvar{x}\typtyped\synvar{\tau}
}

\inferrule[TypLetMut]{
\typctx{\Gamma}\typsc{\typctx{\Delta}}\typderive\synvar{e}\typtyped\synvar{\tau}
}{
\typctx{\Gamma}\typsc\typctx{\Delta}\typderive\synlet\synmut\synvar{x}\syntyped\synvar{\tau}\syneq\synvar{e}\typtyped\synunitt\typproduce\typctx{\Gamma}\typcomma\synvar{x}\typtyped\synvar{\tau}
}
\end{mathpar}
\begin{mathpar}
\inferrule[TypReassign]{
\typctx{\Gamma}\typcomma\synvar{x}\typtyped\synvar{\tau}\typsc\typctx{\Delta}\typderive\synvar{e}\typtyped\synvar{\tau}
}{
\typctx{\Gamma}\typcomma\synvar{x}\typtyped\synvar{\tau}\typsc\typctx{\Delta}\typderive\synvar{x}\syneq\synvar{e}\typtyped\synunitt\typproduce\typctx{\Gamma}\typcomma\synvar{x}\typtyped\synvar{\tau}
}
\end{mathpar}
\begin{mathpar}
\inferrule[TypArrayAssign]{
\typctx{\Gamma}\typsc\synvar{t}\typarrow\typlsquare\synvar{\mu}\typsc n\typrsquare\typcomma\typctx{\Delta}\typderive\synvar{e_1}\typtyped\synvar{t}\\
\typctx{\Gamma}\typsc\synvar{t}\typarrow\typlsquare\synvar{\mu}\typsc n\typrsquare\typcomma\typctx{\Delta}\typderive\synvar{e_2}\typtyped\synint\\
\typctx{\Gamma}\typsc\synvar{t}\typarrow\typlsquare\synvar{\mu}\typsc n\typrsquare\typcomma\typctx{\Delta}\typderive\synvar{e_3}\typtyped\synvar{\mu}\\
}{
\typctx{\Gamma}\typsc\synvar{t}\typarrow\typlsquare\synvar{\mu}\typsc n\typrsquare\typcomma\typctx{\Delta}\typderive \synvar{e_1}\synlsquare\synvar{e_2}\synrsquare\syneq\synvar{e_3}\typtyped\synunitt\typproduce\typctx{\Gamma}
}
\end{mathpar}
\begin{mathpar}
\inferrule[TypSeqAssign]{
\typctx{\Gamma}\typsc\typctx{\Delta}\typderive\synvar{e_1}\typtyped\synseq\synlangle\synvar{\mu}\synrangle\\
\typctx{\Gamma}\typsc\typctx{\Delta}\typderive\synvar{e_2}\typtyped\synint\\
\typctx{\Gamma}\typsc\typctx{\Delta}\typderive\synvar{e_3}\typtyped\synvar{\mu}\\
}{
\typctx{\Gamma}\typsc\typctx{\Delta}\typderive \synvar{e_1}\synlsquare\synvar{e_2}\synrsquare\syneq\synvar{e_3}\typtyped\synunitt\typproduce\typctx{\Gamma}
}
\end{mathpar}
\begin{mathpar}
\inferrule[TypIfThen]{
\typctx{\Gamma}\typsc\typctx{\Delta}\typderive\synvar{e_1}\syntyped\synbool\\
\typctx{\Gamma}\typsc\typctx{\Delta}\typderive\synvar{s_1}\synscellipsis\synvar{s_n}\typtyped\synunitt
}{
\typctx{\Gamma}\typsc\typctx{\Delta}\typderive\synif\synvar{e_1}\synthen\synlbracket\synvar{s_1}\synscellipsis\synvar{s_n}\synrbracket\typtyped\synunitt\typproduce\typctx{\Gamma}
}
\end{mathpar}
\begin{mathpar}
\inferrule[TypIfThenElse]{
\typctx{\Gamma}\typsc\typctx{\Delta}\typderive\synvar{e_1}\syntyped\synbool\\
\typctx{\Gamma}\typsc\typctx{\Delta}\typderive\synvar{s_1}\synscellipsis\synvar{s_n}\typtyped\synunitt\\
\typctx{\Gamma}\typsc\typctx{\Delta}\typderive\synvar{s'_1}\synscellipsis\synvar{s'_n}\typtyped\synunitt
}{
\typctx{\Gamma}\typsc\typctx{\Delta}\typderive\synif\synvar{e_1}\synthen\synlbracket\synvar{s_1}\synscellipsis\synvar{s_n}\synrbracket\synelse\synlbracket\synvar{s'_1}\synscellipsis\synvar{s'_n}\synrbracket\typtyped\synunitt\typproduce\typctx{\Gamma}
}
\end{mathpar}
\begin{mathpar}
\inferrule[TypForLoop]{
\typctx{\Gamma}\typsc\typctx{\Delta}\typderive\synvar{e_1}\typtyped\synint\\
\typctx{\Gamma}\typsc\typctx{\Delta}\typderive\synvar{e_2}\typtyped\synint\\
\typctx{\Gamma}\typcomma\synvar{x}\typtyped\synint\typsc\typctx{\Delta}\typderive\synvar{s_1}\synscellipsis\synvar{s_n}\typtyped\synunitt
}{
\typctx{\Gamma}\typsc\typctx{\Delta}\typderive\synfor\synvar{x}\synin\synvar{e_1}\synrange\synvar{e_2}\;\synlbracket\synvar{s_1}\synscellipsis\synvar{s_n}\synrbracket\typtyped\synunitt\typproduce\typctx{\Gamma}
}

\inferrule[TypExpToStmt]{
\typctx{\Gamma}\typsc\typctx{\Delta}\typderive\synvar{e}\typtyped\synvar{\tau}
}{
\typctx{\Gamma}\typsc\typctx{\Delta}\typderive\synvar{e}\typtyped\synvar{\tau}\typproduce\typctx{\Gamma}
}
\end{mathpar}
\begin{mathpar}
\inferrule[TypSequence]{
\typctx{\Gamma}\typsc\typctx{\Delta}\typderive\synvar{s_1}\typtyped\synvar{\tau'}\typproduce\typctx{\Gamma'}\\
\typctx{\Gamma'}\typsc\typctx{\Delta}\typderive\synvar{s_2}\synscellipsis\synvar{s_n}\typtyped\synvar{\tau}
}{
\typctx{\Gamma}\typsc\typctx{\Delta}\typderive\synvar{s_1}\synscellipsis\synvar{s_n}\typtyped\synvar{\tau}
}

\inferrule[TypSequenceOne]{
\typctx{\Gamma}\typsc\typctx{\Delta}\typderive\synvar{s_1}\typtyped\synvar{\tau}\typproduce\typctx{\Gamma'}
}{
\typctx{\Gamma}\typsc\typctx{\Delta}\typderive\synvar{s_1}\typtyped\synvar{\tau}
}
\end{mathpar}
\end{center}

A \rustspec{} program is a list of items \synvar{i}. Their typing judgment produces both a new \typctx{\Gamma} and \typctx{\Delta}, because an item introduces either a new function or a new
named type. Please note, as mentionned before, that the return type of functions is restricted to
\synvar{\mu}, as returning a reference is forbidden.

\begin{center}
\begin{longtable}{lc}\toprule
Item typing&\typctx{\Gamma}\typsc\typctx{\Delta}\typderive\synvar{i}\typproduce\typctx{\Gamma'}\typsc\typctx{\Delta'}
\\\bottomrule
\end{longtable}
\begin{mathpar}
\inferrule[TypArrayDecl]{
{}
}{
\typctx{\Gamma}\typsc\typctx{\Delta}\typderive\synarraymacro\synlparen\synvar{t}\syncomma\synvar{\mu}\syncomma n\synrparen\typproduce\typctx{\Gamma}\typsc\typctx{\Delta}\typcomma\synvar{t}\typarrow\typlsquare\synvar{\mu}\typsc n\typrsquare
}
\end{mathpar}
\begin{mathpar}
\inferrule[TypFnDecl]{
\typctx{\Gamma}\typcomma\synvar{x_1}\typtyped\synvar{\tau_1}\typellipsis\synvar{x_n}\typtyped\synvar{\tau_n}\typsc\typctx{\Delta}\typderive\synvar{s_1}\synellipsis\synvar{s_m}\typtyped\synvar{\mu}
}{
\typctx{\Gamma}\typsc\typctx{\Delta}\typderive
\synfn\synvar{f}\synlparen\synvar{x_1}\syntyped\synvar{\tau_1}\synellipsis\synvar{x_n}\syntyped\synvar{\tau_n} \synrparen\synarrow\synvar{\mu}\;\synlbracket\synvar{s_1}\synellipsis\synvar{s_m}\synrbracket\typproduce
\typctx{\Gamma}\typcomma\synvar{f}\typtyped\typlparen\synvar{\tau_1}\typellipsis\synvar{\tau_n}\typrparen\typarrow\synvar{\mu}\typsc\typctx{\Delta}
}
\end{mathpar}
\end{center}

\bibliographystyle{plain}
\bibliography{rustspec.bib}

\end{document}
